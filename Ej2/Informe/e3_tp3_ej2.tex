\documentclass[../../e3_tp3_main.tex]{subfiles}

\begin{document}




%capítulo
\chapter{Ejercicio 2}

The inputs of the system are $I$ and $S$, and the outputs are $B_1$ and $B_2$. If $B_1=1$ pump 1 is turned on, and if $B_2$ pump 1 is turned off. The same relation exists between $B_2$ and pump 2.

\todo[inline]{State inputs outputs and future states}
\section{Moore-type FSM implementation}
\subsection{FSM flow-chart}
%\begin{equation}
%	\tikzfig{moore_machine}
%\end{equation}

\subsection{State descriptions}



\subsection{State transition table}

\begin{table}[H]	%moore state descriptions
	\centering
	\begin{tabular}{|c|c|c|c|}
	\hline 
	$y_0,y_1,y_2$/ W & 0 & 1 & output\\ 
	\hline 
	000 & 001 & 000 & 0\\ 
	\hline 
	001 & 011 & 000 & 0\\ 
	\hline 
	011 & 010 & 000 & 0\\ 
	\hline 
	010 & 110 & 000 & 0\\ 
	\hline 
	110 & 001 & 000 & 1\\ 
	\hline 
	\end{tabular} 
	\caption{Possible states for the Moore-type FSM implementation.}
	\label{tab:ej3_moore_states}
\end{table}


\subsection{Outputs and next states Karnaugh maps}

\subsection{Outputs and next states schematics}
%schematics as sum-of-products


\section{Mealy-type FSM implementation}
\subsection{FSM flow-chart}
%\begin{equation}
%	\tikzfig{mealy_machine}
%\end{equation}


\subsection{State descriptions}


\subsection{State transition table}

\begin{table}[H]	%moore state descriptions
	\centering
		\begin{tabular}{|c|c|c|}
		\hline 
		$y_0,y_1$/W & 0 & 1 \\ 
		\hline 
		00 & 01/0 & 00/0 \\ 
		\hline 
		01 & 11/0 & 00/0 \\ 
		\hline 
		11 & 10/0 & 00/0 \\ 
		\hline 
		10 & 00/1 & 00/0 \\ 
		\hline 
		\end{tabular} 
	\caption{Possible states for the Mealy-type FSM implementation.}
	\label{tab:ej3_moore_states}
\end{table}


\subsection{Outputs and next states Karnaugh maps}


\subsection{Output and next states schematics}
%schematics as sum-of-products



\end{document}
