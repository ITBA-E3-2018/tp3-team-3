\documentclass[../../e3_tp3_main.tex]{subfiles}

\begin{document}




%capítulo
\chapter{}



The inputs of the system are $A$, which feeds the FSM with the sequence 1-1-0-1 sequentially and $B$ which corresponds to the current bit of the sequence to compare. To create the sequential flow of $A$, 1-1-0-1 can be loaded in parallel to a shift register and then be shifted. A new auxiliary input $W$ which is simply an XOR of $A$ an $B$, so $W=0$ if $A=B$ and $W=1$ otherwise. This helps reduce the complexity of both FSM's and allows them to be used for to compare against any 4-bit sequence.

The output is $Z=1$ if the sequence has been detected in the last clock cycle, and $Z=0$ otherwise.

When a conclusion is reached (ie. when the input sequence is found to be the same as 1-1-0-1 or different) the FSM jumps to its first state.

\section{Moore-type FSM implementation}
\subsection{FSM flow-chart}

\begin{figure}[H]
	\centering
	\begin{tikzpicture}
	\begin{pgfonlayer}{nodelayer}
		\node [style=estado] (0) at (-4, 0) {1 match};
		\node [style=estado] (1) at (-8, 0) {0 match};
		\node [style=estado] (2) at (4, 0) {3 match};
		\node [style=none] (3) at (-8, 0.75) {};
		\node [style=none] (4) at (-8.75, 0.75) {};
		\node [style=none] (5) at (-7.25, 0.75) {};
		\node [style=none] (6) at (-4.5, 0.75) {};
		\node [style=none] (7) at (-3.5, 0.75) {};
		\node [style=none] (8) at (3.5, 0.75) {};
		\node [style=none] (10) at (-7.25, -1.25) {};
		\node [style=none] (11) at (-4, -0.75) {};
		\node [style=none] (12) at (4, -0.75) {};
		\node [style=none] (13) at (-8.5, 1.5) {W=0};
		\node [style=estado] (14) at (0, 0) {2 match};
		\node [style=none] (15) at (-0.5, 0.75) {};
		\node [style=none] (16) at (0.5, 0.75) {};
		\node [style=none] (17) at (0, -0.75) {};
		\node [style=none] (18) at (-8, -0.75) {};
		\node [style=none] (19) at (-6, 1.5) {W=0};
		\node [style=none] (20) at (-2, 1.5) {W=0};
		\node [style=none] (21) at (2.25, 1.525) {W=0};
		\node [style=none] (23) at (-7.5, -1.75) {W=1};
		\node [style=estado] (24) at (8, 0) {4 match};
		\node [style=none] (25) at (7.5, 0.75) {};
		\node [style=none] (26) at (4.5, 0.75) {};
		\node [style=none] (27) at (6.25, 1.525) {W=0};
		\node [style=none] (28) at (8, -0.75) {};
		\node [style=none] (29) at (8.5, 0.75) {};
		\node [style=none] (30) at (-4, 0.75) {};
		\node [style=none] (31) at (2, 4) {W=0};
	\end{pgfonlayer}
	\begin{pgfonlayer}{edgelayer}
		\draw [style=transicion, bend left=75, looseness=1.25] (4.center) to (3.center);
		\draw [style=transicion, bend left, looseness=0.50] (5.center) to (6.center);
		\draw [style=transicion, bend left, looseness=0.50] (7.center) to (15.center);
		\draw [style=transicion, bend left, looseness=0.50] (16.center) to (8.center);
		\draw [style=transicion, bend left] (12.center) to (10.center);
		\draw [style=transicion, bend left] (17.center) to (10.center);
		\draw [style=transicion, bend left, looseness=0.75] (11.center) to (10.center);
		\draw [style=transicion, bend right, looseness=0.25] (10.center) to (18.center);
		\draw [style=transicion, bend left, looseness=0.50] (26.center) to (25.center);
		\draw [style=transicion, bend right=90, looseness=0.75] (29.center) to (30.center);
		\draw [style=transicion, bend left] (28.center) to (10.center);
	\end{pgfonlayer}
\end{tikzpicture}

	\caption{Flow diagram for the Moore-type FSM implementation.}
\end{figure}
\subsection{State descriptions}
\begin{table}[H]	%moore state descriptions
	\centering
		\begin{tabular}{|c|c|c|c|}
		\hline 
		$y_0,y_1,y_2$ & State name & Description & Z \\ 
		\hline 
		000 & 0 MATCH & No bits were matched & 0\\ 
		\hline 
		001 & 1 MATCH & Only one bit was matched & 0\\ 
		\hline 
		011 & 2 MATCH & Only two bits were matched & 0\\ 
		\hline 
		010 & 3 MATCH & Only three bits were matched & 0\\ 
		\hline 
		110 & 4 MATCH & The whole sequence has been matched & 1\\ 
		\hline 
		\end{tabular} 
	\caption{Possible states for the Moore-type FSM implementation.}
	\label{tab:ej2_moore_states}
\end{table}


\subsection{State transition table}

\begin{table}[H]	%moore state descriptions
	\centering
	\begin{tabular}{|c|c|c|c|}
	\hline 
	$y_0,y_1,y_2$/ W & 0 & 1 & output\\ 
	\hline 
	000 & 001 & 000 & 0\\ 
	\hline 
	001 & 011 & 000 & 0\\ 
	\hline 
	011 & 010 & 000 & 0\\ 
	\hline 
	010 & 110 & 000 & 0\\ 
	\hline 
	110 & 001 & 000 & 1\\ 
	\hline 
	\end{tabular} 
	\caption{Possible states for the Moore-type FSM implementation.}
	\label{tab:ej3_moore_states}
\end{table}


\subsection{Outputs and next states Karnaugh maps}
\begin{figure}[H]
	\centering
	\includegraphics[scale=0.65]{figures/e3_tp3_ej2_moore_y0_kmap.jpg}
	\caption{$Y_0$ Karnaugh map}
\end{figure}

\begin{figure}[H]
	\centering
	\includegraphics[scale=1]{figures/e3_tp3_ej2_moore_y1_kmap.jpg}
	\caption{$Y_1$ Karnaugh map}
\end{figure}

\begin{figure}[H]
	\centering
	\includegraphics[scale=1]{figures/e3_tp3_ej2_moore_z_kmap.jpg}
	\caption{$Z$ Karnaugh map}
\end{figure}


\subsection{Outputs and next states schematics}
\begin{figure}[H]
	\centering
	\includegraphics{figures/ej2_Y0_schem.PNG}
	\caption{$Y_0$ schematic as a sum-of-products}
\end{figure}
\begin{figure}[H]
	\centering
	\includegraphics{figures/ej2_Y1_schem.PNG}
	\caption{$Y_1$ schematic as a sum-of-products}
\end{figure}
\begin{figure}[H]
	\centering
	\includegraphics{figures/ej2_Y2_schem.PNG}
	\caption{$Y_2$ schematic as a sum-of-products}
\end{figure}



\section{Mealy-type FSM implementation}
\subsection{FSM flow-chart}
\begin{figure}[H]
	\centering
	\begin{tikzpicture}
	\begin{pgfonlayer}{nodelayer}
		\node [style=estado] (0) at (-6.5, 2) {Pump 1 only};
		\node [style=estado] (1) at (6.5, 2) {Next single pump: 2};
		\node [style=estado] (3) at (-6.5, -2.325) {Next single pump: 1};
		\node [style=estado] (4) at (6.5, -2.325) {Pump 2 only};
		\node [style=none] (5) at (6.5, 1.25) {};
		\node [style=none] (6) at (6.5, -1.575) {};
		\node [style=none] (7) at (6.5, -1.575) {};
		\node [style=none] (8) at (-6.5, 1.25) {};
		\node [style=none] (9) at (-6.5, -1.575) {};
		\node [style=none] (19) at (0, -4.75) {I=S=1 / A=B=0};
		\node [style=none] (20) at (0, -3.25) {I=S=0 / A=B=1};
		\node [style=none] (25) at (-5, 0.25) {I=1, S=0};
		\node [style=none] (26) at (5.25, 0.25) {I=1, S=0};
		\node [style=none] (28) at (-5, -0.575) {A=1, B=0};
		\node [style=none] (30) at (5.25, -0.575) {A=0, B=1};
		\node [style=none] (35) at (7.25, -3) {};
		\node [style=none] (36) at (5, -5) {};
		\node [style=none] (37) at (5.75, -3) {};
		\node [style=none] (38) at (5, -3.625) {};
		\node [style=none] (39) at (-5, -5.05) {};
		\node [style=none] (40) at (-7, -2.9) {};
		\node [style=none] (41) at (-5, -3.65) {};
		\node [style=none] (42) at (-5.625, -2.9) {};
		\node [style=none] (43) at (0, 5.25) {I=S=1 / A=B=0};
		\node [style=none] (44) at (0, 3.75) {I=S=0 / A=B=1};
		\node [style=none] (45) at (-7.25, 2.75) {};
		\node [style=none] (46) at (-5, 4.75) {};
		\node [style=none] (47) at (-5.75, 2.75) {};
		\node [style=none] (48) at (-5, 3.375) {};
		\node [style=none] (49) at (5, 4.8) {};
		\node [style=none] (50) at (7, 2.65) {};
		\node [style=none] (51) at (5, 3.4) {};
		\node [style=none] (52) at (5.625, 2.65) {};
	\end{pgfonlayer}
	\begin{pgfonlayer}{edgelayer}
		\draw [style=transicion] (9.center) to (8.center);
		\draw [style=transicion] (5.center) to (7.center);
		\draw [style=transicion, bend left=45] (35.center) to (36.center);
		\draw [style=transicion, bend left=45] (37.center) to (38.center);
		\draw [style=transicion, bend left=45] (39.center) to (40.center);
		\draw [style=transicion, bend left=45] (41.center) to (42.center);
		\draw [style=transicion] (36.center) to (39.center);
		\draw [style=transicion] (38.center) to (41.center);
		\draw [style=transicion, bend left=45] (45.center) to (46.center);
		\draw [style=transicion, bend left=45] (47.center) to (48.center);
		\draw [style=transicion, bend left=45] (49.center) to (50.center);
		\draw [style=transicion, bend left=45, looseness=1.25] (51.center) to (52.center);
		\draw [style=transicion] (46.center) to (49.center);
		\draw [style=transicion] (48.center) to (51.center);
	\end{pgfonlayer}
\end{tikzpicture}

	\caption{Flow diagram for the Mealy-type FSM implementation.}
\end{figure}


\subsection{State descriptions}
\begin{table}[H]	%moore state descriptions
	\centering
		\begin{tabular}{|c|c|c|c|}
		\hline 
		$y_0,y_1$ & State name & Description \\ 
		\hline 
		00 & 0 MATCH & No bits were matched \\ 
		\hline 
		01 & 1 MATCH & Only one bit was matched\\ 
		\hline 
		11 & 2 MATCH & Only two bits were matched\\ 
		\hline 
		10 & 3 MATCH & Only three bits were matched \\ 
		\hline
		\end{tabular} 
	\caption{Possible states for the Moore-type FSM implementation. Note: when the sequence has been found, the FSM transitions from 3 MATCH to 0 MATCH.}
	\label{tab:ej2_mealy_states}
\end{table}



\subsection{State transition table}

\begin{table}[H]	%moore state transitions
	\centering
		\begin{tabular}{|c|c|c|}
		\hline 
		$y_0,y_1$/W & 0 & 1 \\ 
		\hline 
		00 & 01/0 & 00/0 \\ 
		\hline 
		01 & 11/0 & 00/0 \\ 
		\hline 
		11 & 10/0 & 00/0 \\ 
		\hline 
		10 & 00/1 & 00/0 \\ 
		\hline 
		\end{tabular} 
	\caption{Possible states for the Mealy-type FSM implementation.}
	\label{tab:ej2_mealy_state_transitions}
\end{table}


\subsection{Outputs and next states Karnaugh maps}
\begin{figure}[H]
	\centering
	\includegraphics[scale=1]{figures/e3_tp3_ej2_mealy_y0_kmap.jpg}
	\caption{$Y_0$ Karnaugh map}
\end{figure}

\begin{figure}[H]
	\centering
	\includegraphics[scale=1]{figures/e3_tp3_ej2_mealy_y1_kmap.jpg}
	\caption{$Y_1$ Karnaugh map}
\end{figure}

\begin{figure}[H]
	\centering
	\includegraphics[scale=0.9]{figures/e3_tp3_ej2_mealy_z_kmap.jpg}
	\caption{$Z$ Karnaugh map}
\end{figure}

\subsection{Output and next states schematics}
\begin{figure}[H]
	\centering
	\includegraphics{figures/ej2_Y0_schem_mealy.PNG}
	\caption{$Y_0$ schematic as a sum-of-products}
\end{figure}
\begin{figure}[H]
	\centering
	\includegraphics{figures/ej2_Y1_schem_mealy.PNG}
	\caption{$Y_1$ schematic as a sum-of-products}
\end{figure}
\begin{figure}[H]
	\centering
	\includegraphics{figures/ej2_Z_schem_mealy.PNG}
	\caption{$Z$ schematic as a sum-of-products}
\end{figure}



\end{document}
